% Options for packages loaded elsewhere
\PassOptionsToPackage{unicode}{hyperref}
\PassOptionsToPackage{hyphens}{url}
%
\documentclass[
]{article}
\usepackage{amsmath,amssymb}
\usepackage{lmodern}
\usepackage{iftex}
\ifPDFTeX
  \usepackage[T1]{fontenc}
  \usepackage[utf8]{inputenc}
  \usepackage{textcomp} % provide euro and other symbols
\else % if luatex or xetex
  \usepackage{unicode-math}
  \defaultfontfeatures{Scale=MatchLowercase}
  \defaultfontfeatures[\rmfamily]{Ligatures=TeX,Scale=1}
\fi
% Use upquote if available, for straight quotes in verbatim environments
\IfFileExists{upquote.sty}{\usepackage{upquote}}{}
\IfFileExists{microtype.sty}{% use microtype if available
  \usepackage[]{microtype}
  \UseMicrotypeSet[protrusion]{basicmath} % disable protrusion for tt fonts
}{}
\makeatletter
\@ifundefined{KOMAClassName}{% if non-KOMA class
  \IfFileExists{parskip.sty}{%
    \usepackage{parskip}
  }{% else
    \setlength{\parindent}{0pt}
    \setlength{\parskip}{6pt plus 2pt minus 1pt}}
}{% if KOMA class
  \KOMAoptions{parskip=half}}
\makeatother
\usepackage{xcolor}
\usepackage[margin=1in]{geometry}
\usepackage{graphicx}
\makeatletter
\def\maxwidth{\ifdim\Gin@nat@width>\linewidth\linewidth\else\Gin@nat@width\fi}
\def\maxheight{\ifdim\Gin@nat@height>\textheight\textheight\else\Gin@nat@height\fi}
\makeatother
% Scale images if necessary, so that they will not overflow the page
% margins by default, and it is still possible to overwrite the defaults
% using explicit options in \includegraphics[width, height, ...]{}
\setkeys{Gin}{width=\maxwidth,height=\maxheight,keepaspectratio}
% Set default figure placement to htbp
\makeatletter
\def\fps@figure{htbp}
\makeatother
\setlength{\emergencystretch}{3em} % prevent overfull lines
\providecommand{\tightlist}{%
  \setlength{\itemsep}{0pt}\setlength{\parskip}{0pt}}
\setcounter{secnumdepth}{-\maxdimen} % remove section numbering
\ifLuaTeX
  \usepackage{selnolig}  % disable illegal ligatures
\fi
\IfFileExists{bookmark.sty}{\usepackage{bookmark}}{\usepackage{hyperref}}
\IfFileExists{xurl.sty}{\usepackage{xurl}}{} % add URL line breaks if available
\urlstyle{same} % disable monospaced font for URLs
\hypersetup{
  hidelinks,
  pdfcreator={LaTeX via pandoc}}

\author{}
\date{\vspace{-2.5em}}

\begin{document}

V1.1\_20221021
\includegraphics{RackMultipart20221220-1-9i7noz_html_c17ef1b6749dd55a.png}

\hypertarget{user-guide}{%
\section{User guide}\label{user-guide}}

\textbf{\protect\hyperlink{_up3x7w38quym}{Welcome to DROPS} 2}

\protect\hyperlink{_paicald3w2xf}{What is DROPS?} 2

\protect\hyperlink{_6eanlztnrvpa}{Which data can be requested through
DROPS?} 2

\protect\hyperlink{_wqj08dx9lskp}{Who runs DROPS?} 2

\protect\hyperlink{_6pnjc4fh9u9i}{Why do we use DROPS?} 3

\textbf{\protect\hyperlink{_iryi1psurnbs}{How does DROPS work?} 4}

\protect\hyperlink{_w61ki29hp26j}{The DROPS system} 4

\protect\hyperlink{_c0rqoo2n1ad5}{Abstract submission} 6

\protect\hyperlink{_d60pemtq0onl}{Registering your study} 7

\protect\hyperlink{_j9vo6i2gjp6q}{For non-ESM studies} 7

\protect\hyperlink{_9l6plnofve8}{OSF registrations} 7

\protect\hyperlink{_dypi9oxebkkg}{AsPredicted registrations} 7

\protect\hyperlink{_4priwppr2d24}{For ESM studies} 8

\protect\hyperlink{_c7z830chg34n}{OSF registrations} 8

\protect\hyperlink{_7fg39v7hd0te}{AsPredicted registrations} 8

\protect\hyperlink{_qlx2etgrq9hr}{Power analysis} 9

\protect\hyperlink{_rnzjcirxyvgf}{Statistical power} 9

\protect\hyperlink{_bwod4c38xqma}{How to estimate statistical power} 9

\protect\hyperlink{_7h6c6a2xjdz9}{Requesting a subsample of data to
calculate power} 9

\protect\hyperlink{_2dzt9xeysbee}{Variable access request} 10

\protect\hyperlink{_45gpzwmgy9bj}{Receiving data} 10

\protect\hyperlink{_pcdajm75pc8x}{Special cases} 12

\protect\hyperlink{_ffqkpx648yj2}{Registered Reports using DROPS} 12

\protect\hyperlink{_frgb5bv2wcbm}{About Registered Reports} 12

\protect\hyperlink{_6or3tby42f75}{Registered Reports and DROPS} 12

\protect\hyperlink{_q6nxcc4a4n2f}{Exploratory studies} 13

\textbf{\protect\hyperlink{_zc50z372ij74}{Frequently Asked Questions}
13}

\hypertarget{welcome-to-drops}{%
\section{Welcome to DROPS}\label{welcome-to-drops}}

\hypertarget{what-is-drops}{%
\subsection{What is DROPS?}\label{what-is-drops}}

Data cuRation for Open Science (DROPS) is a data checkout system. It
allows researchers to request data from several datasets collected by
the Center of Contextual Psychiatry at KU Leuven and tracks who
requested and accessed which data, and when, in a detailed way. In this
manual, you can find an overview of the steps necessary for obtaining
data from the DROPS system.

\hypertarget{which-data-can-be-requested-through-drops}{%
\subsection{Which data can be requested through
DROPS?}\label{which-data-can-be-requested-through-drops}}

As of now, you can use DROPS to request data from the four following
studies:

\begin{itemize}
\item
  \href{https://gbiomed.kuleuven.be/english/research/50000666/50000673/cpp/research-1/social-interaction/sigma/index.htm}{\emph{SIGMA}}
  --⁠ a longitudinal study that combined questionnaires and ecological
  momentary assessment data to assess the relationships between social
  risk factors, psychological mechanisms and psychopathology in an
  initial sample of 1,913 adolescents. So far, the data from the first
  and second wave, as well as the additional COVID wave are available in
  DROPS. The contact persons for the SIGMA dataset are
  \href{https://www.kuleuven.be/wieiswie/nl/person/00115655}{Robin
  Achterhof} and
  \href{https://www.kuleuven.be/wieiswie/nl/person/00117419}{Olivia
  Kirtley}. The protocol for the SIGMA study can be found
  \href{https://psyarxiv.com/jp2fk/}{here}\href{https://www.zotero.org/google-docs/?agskoG}{(Kirtley,
  Achterhof, et al., 2021)}. \href{https://osf.io/xwvc5/}{Click here}to
  access an Open Science Framework repository with information about the
  studies published using SIGMA data and the measures included in the
  study.
\item
  \href{https://pubmed.ncbi.nlm.nih.gov/31878966/}{\emph{Interact}} -- a
  study aimed at evaluating the efficacy of Acceptance and Commitment
  Therapy in Daily Life, an ecological momentary intervention. Broadly,
  the dataset contains variables related to psychotic experiences,
  psychopathology, and global and social functioning. The contact person
  for the Interact study is
  \href{https://www.kuleuven.be/wieiswie/nl/person/00106722}{Thomas}\href{https://www.kuleuven.be/wieiswie/nl/person/00106722}{Vaessen}.
\item
  \emph{EXPOSS} -- A (non-ESM) pilot study of 371 first-year Psychology
  students to develop the exposure to self-harm and suicide scale
  (EXPOSS; Kirtley, De Roover, \& Dumon, 2020), a new measure of direct
  and media exposure to self-harm and suicide. Additional variables
  include depression, anxiety, stress, defeat, entrapment, social
  support, negative life events, positive and negative affect, and
  attitudes towards data sharing. The contact person for the EXPOSS
  dataset is
  \href{https://www.kuleuven.be/wieiswie/nl/person/00117419}{Olivia
  Kirtley}.
\item
  The \_Methodology\_dataset contains the data used for a project that
  investigated various methodological aspects of ESM research, including
  the effect of ESM questionnaire length and sampling frequency on
  participant burden on data quality (paper available
  \href{https://journals.sagepub.com/doi/10.1177/1073191120957102}{here};
  \href{https://www.zotero.org/google-docs/?UNkfRy}{Eisele et al.,
  2020)}, affective structure and measurement invariance (paper
  available
  \href{https://www.sciencedirect.com/science/article/pii/S0092656621000313?casa_token=Z6l7hIkmucUAAAAA:VLP3vqtJO_7lx2BKclkfVwFGg3F1LkEuTUvHGo8DLSna_pBUR-pjJVbuMHkbfafEnijEkHmfEcQ}{here};
  \href{https://www.zotero.org/google-docs/?rllNl0}{Eisele et al.,
  2021)}.
\end{itemize}

\hypertarget{who-runs-drops}{%
\subsection{Who runs DROPS?}\label{who-runs-drops}}

The DROPS system is run by the
\href{https://gbiomed.kuleuven.be/english/research/50000666/50000673/cpp}{Cente}\href{https://gbiomed.kuleuven.be/english/research/50000666/50000673/cpp}{r
for Contextual Psychiatry} at KU Leuven. The only person who can access
the datasets and share them with researchers is the data manager,
\href{https://www.kuleuven.be/wieiswie/en/person/00007576}{Dr Martien
Wampers}. Additionally, there is a `triage' committee for each of the
datasets --⁠ one or more researchers that do not by default have the
access to the data, but who keep track of what data have been used by
whom and what studies are in progress using data from a particular
project. The triage committee is in charge of approving abstract
submissions and variable access requests.

\hypertarget{why-do-we-use-drops}{%
\subsection{Why do we use DROPS?}\label{why-do-we-use-drops}}

Using DROPS provides the following benefits to both individual
researchers and science in general:

\begin{itemize}
\tightlist
\item
  \emph{It makes the use of data more efficient.} A lot of time and
  resources goes into collecting data at the Center for Contextual
  Psychiatry. As such, we want to use the data efficiently and make it
  available to other researchers, so that they can use it to answer
  their own research questions.
\item
  \emph{It prevents waste of time.} DROPS saves researchers time by
  preventing studies with identical research questions being conducted
  on the same datasets. If a researcher requests data to investigate a
  set of questions another person/team is already working on, they will
  be notified in early stages of their research process.
\item
  \emph{It facilitates confirmatory approach to secondary data
  analysis.} It is important to be able to show which analytical
  decisions were based on what information. When you plan and register
  your analyses before the data are collected, it is easy to distinguish
  between the analytical decisions that were made prior to and after you
  could see the data. Nonetheless, in secondary data analysis, the line
  between exploratory and confirmatory analysis can get blurry. By
  adding an extra step to the data sharing process, DROPS allows for a
  confirmatory approach to secondary data analysis.
\item
  \emph{It facilitates credibility in research.} Before obtaining the
  data for their secondary analyses, researchers are asked to clearly
  state the aims of their study and develop and pre-register a detailed
  analysis plan. Each step of the data release process is logged in a
  transparent way. As such, the researchers are provided with a series
  of receipts that prove they did not have access to the data before
  analyzing them. This improves the evidential value of secondary data
  analyses, increases the reusability of datasets and facilitates the
  transparency of research.
\end{itemize}

\hypertarget{section}{%
\section{}\label{section}}

\hypertarget{how-does-drops-work}{%
\section{How does DROPS work?}\label{how-does-drops-work}}

\hypertarget{section-1}{%
\subsection{}\label{section-1}}

\hypertarget{the-drops-system}{%
\subsection{The DROPS system}\label{the-drops-system}}

In this section, we will briefly walk you through the process of
obtaining data from DROPS. The whole workflow is also illustrated in
Figure 1 (below), and described in more detail in the following
sections.

The first step of the process is
\protect\hyperlink{_c0rqoo2n1ad5}{drafting and submitting
an}\protect\hyperlink{_c0rqoo2n1ad5}{\textbf{abstract}} (steps 1 and 2
in the flowchart) --⁠ a brief description of your research questions and
analysis plans. Before drafting your abstract, be sure to acquaint
yourself with the codebook of the study you are planning to request data
from, so that you have a clear idea about the sample and available
variables. After your abstract is approved by the relevant triage
committee, you will automatically receive an email with a link to the
\protect\hyperlink{_2dzt9xeysbee}{\textbf{variable}} access request form
(steps 3 and 4), which will prompt you to provide the details of the
variables you require (from the codebook). All researchers wishing to
use data from the CCP are required to register their study on the Open
Science Framework (OSF) or AsPredicted. Therefore you will also be asked
to provide a link to the
\protect\hyperlink{_d60pemtq0onl}{\textbf{registration of your study}}
in the variable access request form (steps 5 and 6). The
post-registration should contain a more detailed version of the
information contained in your abstract. After the triage committee
approves your variable access request, the data manager will
\protect\hyperlink{_kciim8tnzl6e}{send you a
link}\protect\hyperlink{_kciim8tnzl6e}{to access the data}, along with a
confirmation message of data transmission (steps 7 and 8).

It is also possible to use the DROPS system to
\protect\hyperlink{_uquq4hhmcg15}{conduct a study following the
Registered Reports publication format}, to
\protect\hyperlink{_qlx2etgrq9hr}{request a subsample of the data} to
obtain parameter estimates for power analysis or to
\protect\hyperlink{_q6nxcc4a4n2f}{request data for exploratory studies}.
We describe the procedures for these specific situations in more detail
below.

\includegraphics{RackMultipart20221220-1-9i7noz_html_2fcce4a48b3c9234.png}

\emph{Figure 1: A flowchart of the DROPS data request process.}

\hypertarget{section-2}{%
\subsection{}\label{section-2}}

\hypertarget{abstract-submission}{%
\subsection{Abstract submission}\label{abstract-submission}}

The first step in requesting data through DROPS is filling out the DROPS
\href{https://redcap.gbiomed.kuleuven.be/surveys/?s=WDYAFAHWK4}{abstract
submission form}. Together with your co-author team (e.g., your
supervisors and collaborators), you draft an abstract for your study in
which you briefly describe the background rationale (including key
references), clearly state your research questions and hypotheses, and
give a brief overview of the analyses you are planning to conduct. You
will also be asked to provide a precise description of the sample you
want to request from DROPS (e.g.~``all participants aged 12 years
old''), and list all variables you are planning to use.

To make the process of drafting the abstract with your co-authors
easier, we recommend using
\href{https://docs.google.com/document/d/1WmAHdhIU37RJest4ozNItDl_Ygml4BUomt8a_EJ8yNs/edit}{this
template}. After completing the draft, you can easily copy-paste the
text from the template into the corresponding text box in the DROPS
submission form, which will also allow you to indicate which study you
are requesting data from.

After submission, you can download a date and time-stamped overview of
your abstract submission.

\emph{IMPORTANT!} After submitting the abstract in DROPS, it is no
longer possible to make any changes to the abstract. Therefore, please
make sure that the entire abstract is approved by all researchers
involved in the study and that you have thoroughly proofread it before
submitting it to DROPS.

After submission, the triage committee will review the abstract and
decide whether the abstract meets the required criteria:

\begin{itemize}
\tightlist
\item
  the abstract should not have substantial overlap with other planned or
  ongoing work on the same datasets
\item
  the abstract should contain a sufficient amount of details and no
  required information should be missing
\item
  the authors only request variables that are included in the dataset.
\end{itemize}

Abstracts will be either rejected or given a revise and resubmit when
they overlap with ongoing or planned work, or when they contain errors.
In case your abstract is rejected or given a revise and resubmit, you
are welcome to either adapt the abstract accordingly and resubmit, or
submit an abstract for a new study.

\hypertarget{registering-your-study}{%
\subsection{Registering your study}\label{registering-your-study}}

\hypertarget{for-non-esm-studies}{%
\subsubsection{For non-ESM studies}\label{for-non-esm-studies}}

After the approval of your abstract (step 3 in the flowchart), you will
be asked to provide a registration of your study (step 5) as a part of
the variable access request (step 4). In this part of the user guide, we
provide guidance on the paper registration process.

\hypertarget{osf-registrations}{%
\paragraph{OSF registrations}\label{osf-registrations}}

The Open Science Framework (OSF) is a free-to-use platform for
facilitating open science practices. The OSF is one of the most
widely-used platforms to register a research study. Either prior to data
collection (pre-registration) or before accessing and analyzing already
collected data (post-registration), you can register your research
questions and analytical plans and receive a time and date-stamped
confirmation of this. If you want to learn more about the reasons for
and the benefits of registration, we recommend
\href{https://www.pnas.org/doi/epdf/10.1073/pnas.1708274114}{this
paper}\href{https://www.zotero.org/google-docs/?E8wJIv}{(Nosek et al.,
2018)}. Below, you can find more resources on how to register your
studies on OSF and how to register secondary data analyses. For more
information about the differences between pre-registration and
post-registration, we refer the reader to
\href{https://psycnet.apa.org/doiLanding?doi=10.1037\%2Fabn0000451}{this
paper about the registration
continuum}\href{https://www.zotero.org/google-docs/?z1Lw8r}{(Benning et
al., 2019)}.

Please note that while the resources mentioned above often talk about
\_pre-\_registration, we refer to \_post-\_registration in this user
guide. Similar to pre-registration, post-registration is the practice of
registering your hypotheses and analysis plans prior to accessing the
data and conducting any analyses; in post-registration, however, the
data are already collected at the time of the registration.

For more information about the registration of studies using preexisting
data, please refer to the
\href{https://open.lnu.se/index.php/metapsychology/article/view/2625}{template
and tutorial} by \href{https://www.zotero.org/google-docs/?wZgHX5}{Van
den Akker et al.~(2021)}.

When you start a new registration on OSF, there is now an option for
``secondary data registration''.

More information on how to register a study on the OSF:
\url{https://help.osf.io/hc/en-us/articles/360019738834-Create-a-Preregistration}

More information on how to make a view-only link to the registration:
\url{https://help.osf.io/hc/en-us/articles/360042097853-Create-a-View-only-Link-for-a-Registration}

General OSF help page about registrations:
\url{https://help.osf.io/hc/en-us/categories/360001550953-Registrations}

\hypertarget{aspredicted-registrations}{%
\paragraph{AsPredicted registrations}\label{aspredicted-registrations}}

If you are requesting to use data for a non-ESM study, you can also
register your study on the AsPredicted site (aspredicted.com). This is
an easy and brief way to pre-register your study and make your study
registration available for others to read. To pre-register your study at
AsPredicted, you are asked to answer nine short questions about the
research plan, hypotheses and analyses.

If you intend to request data from DROPS, we advise that you answer
Question 1 \emph{``Have any data been collected for this study
already?''} of the AsPredicted template by selecting the following
option:

``\emph{It's complicated. We have already collected some data but
explain in Question 8 why readers may consider this a valid
pre-registration nevertheless.''} In Question 8, you can then briefly
explain that you are planning to use pre-existing data which you will
access through a data-access system which time-stamps your access.
Similarly to OSF, after registration, you can also create an anonymous
version (PDF) of your registration for e.g.~reviewers if you do not want
to make your registration public yet.

\hypertarget{section-3}{%
\subsubsection{}\label{section-3}}

\hypertarget{for-esm-studies}{%
\subsubsection{For ESM studies}\label{for-esm-studies}}

\hypertarget{osf-registrations-1}{%
\paragraph{OSF registrations}\label{osf-registrations-1}}

If you are requesting to use ESM data from DROPS, please use the
\href{https://osf.io/2chmu/}{registration template for ESM
research}\href{https://www.zotero.org/google-docs/?ORGIOC}{(Kirtley,
Lafit, et al., 2021)}. The template is adapted from the original
\href{https://osf.io/x5w7h/}{Pre-Registration Challenge} (Mellor et al.,
2019) and
\href{https://open.lnu.se/index.php/metapsychology/article/view/2625}{pre-registration
of pre-existing
data}\href{https://www.zotero.org/google-docs/?EKoiMH}{(Van den Akker et
al., 2021)} templates. The template includes questions that have been
developed or adapted to be specifically relevant for ESM studies. This
OSF page also includes two examples of how to complete the template,
including a pre-registration (registering a study prior to data
collection; Example 1) and a post-registration (registering a study of
already collected data prior to accessing and analyzing them; Example
2).

For detailed instructions on using the template, please refer to the
\href{https://journals.sagepub.com/doi/abs/10.1177/2515245920924686}{tutorial
paper}\href{https://www.zotero.org/google-docs/?pBAwde}{(Kirtley, Lafit,
et al., 2021)}.

If you require additional information about any of the studies in DROPS
to complete your pre-registration, we advise that you contact the DROPS
data manager,
\href{https://www.kuleuven.be/wieiswie/en/person/00007576}{Dr Martien
Wampers}. She will provide you with more details.

\hypertarget{aspredicted-registrations-1}{%
\paragraph{AsPredicted
registrations}\label{aspredicted-registrations-1}}

For those less familiar with pre-registration (e.g., master students)
who do not wish to pre-register their study at OSF, we recommend the
shorter registration at AsPredicted. If you decide to use AsPredicted
for your registration, please use Question 3: ``\emph{Describe the key
dependent variable(s) specifying how they will be measured''} to
indicate which of the variables you are requesting are ESM variables and
which are non-ESM variables. We also advise that when describing your
ESM variables, you explain if and how any transformation or indices such
as sum scores or averages will be constructed from them.

\hypertarget{power-analysis}{%
\subsubsection{Power analysis}\label{power-analysis}}

\hypertarget{statistical-power}{%
\paragraph{Statistical power}\label{statistical-power}}

In the process of registering your secondary data analysis, you will be
asked to specify the statistical power of your planned analyses. Put
simply, statistical power is the probability of rejecting the null
hypothesis in your study, given that there is a real effect present
\href{https://www.zotero.org/google-docs/?xgNmjx}{(i.e., the null
hypothesis is
false}\href{https://www.zotero.org/google-docs/?xgNmjx}{;}\href{https://www.zotero.org/google-docs/?xgNmjx}{Cohen,
1992)}. Conducting studies with only a small probability of finding an
effect when it actually exists can result in a waste of time and
resources, and it makes even significant effects harder to interpret. As
such, there is a push for psychologists to base their sample size plans
on a priori estimation of statistical power. However, given the complex
nature of the analyses used for ESM data, only about 2\% of ESM studies
on psychopathology report a power calculation
\href{https://www.zotero.org/google-docs/?eipiiN}{(Trull \&
Ebner-Priemer, 2020)}.

We require registrations of (confirmatory) studies to include an a
priori power analysis in order to contribute to the robustness of mental
health research, make the interpretation of results straightforward, and
prevent resource wastage.

\hypertarget{how-to-estimate-statistical-power}{%
\paragraph{How to estimate statistical
power}\label{how-to-estimate-statistical-power}}

Of course, given that the data included in DROPS is already collected,
you cannot influence the available sample size. Still, it is very useful
to have an idea about the probability of your study in finding the
smallest effect size of interest, and potentially adjusting your
analytical plans (e.g., simplifying your statistical models or
designating some research questions as exploratory) if the statistical
power is insufficient.

For the analysis of non-ESM data using statistical techniques such as
\emph{t}-tests, ANOVA or regression, we recommend using either the
software
\href{https://www.psychologie.hhu.de/arbeitsgruppen/allgemeine-psychologie-und-arbeitspsychologie/gpower}{G-power}\href{https://www.zotero.org/google-docs/?wcD7sK}{(Faul
et al., 2009)} or the R package
\href{https://cran.r-project.org/web/packages/pwr/pwr.pdf}{\emph{pwr}}\href{https://www.zotero.org/google-docs/?zgqomC}{(Champely
et al., 2018)}\emph{.} For mediation analyses, widely used in mental
health research, power analysis is more complicated
\href{https://www.zotero.org/google-docs/?ZJP55s}{(Perugini et al.,
2018)}: under specific assumptions, analytical solutions are available
in the R package
\href{https://cran.r-project.org/web/packages/powerMediation/index.html}{\emph{powerMediation}}\href{https://www.zotero.org/google-docs/?dHtGlO}{(Qiu,
2021)}; in other scenarios, using simulations is the most feasible way
of assessing statistical power. Similarly, in the case of more complex
multilevel or longitudinal models using ESM data, calculating power is
often only possible via simulation methods. We recommend using the
\href{https://journals.sagepub.com/doi/full/10.1177/2515245920978738}{interactive
Shiny app and accompanying tutorial
paper}\href{https://www.zotero.org/google-docs/?nap5qJ}{(Lafit et al.,
2020)} that makes the simulation approach to power analysis for
(auto-regressive) multilevel models more straightforward.

\hypertarget{requesting-a-subsample-of-data-to-calculate-power}{%
\paragraph{Requesting a subsample of data to calculate
power}\label{requesting-a-subsample-of-data-to-calculate-power}}

Regardless of which approach to power analysis you choose, you will
likely need to have an idea about some characteristics of your data.
Even in the relatively simple multilevel autoregressive(1) model used
for estimating inertia from ESM data, you will need to specify the
standard error of level 1 errors, random slopes and random intercepts.
These parameters can usually not be easily derived from the literature.

As a convenient solution for this, DROPS allows you to request a small
subsample of the dataset you are planning to analyze, to estimate the
parameters necessary for power analysis. The process of obtaining the
subsample is a simplified version of the standard DROPS workflow: you
will still have to submit an abstract (in which you clearly state that
you are only requesting a subsample of the data), but it is not
necessary to upload a registration at this point.

\hypertarget{variable-access-request}{%
\subsection{Variable access request}\label{variable-access-request}}

After the triage committee approves your abstract, you will
automatically receive a link to the variable access request form. You
will then be asked to provide a view-only link to the registration of
your study. In the following parts of this manual, you can find all the
information you need about registering your study and generating
view-only links.

If you are using the Open Science Framework (OSF) to register your
study, please provide a view-only link to the pre-registration itself,
not to the whole OSF project. The view-only link for the OSF project
does not provide access to the pre-registration.

You will also be asked to provide a list of the exact variables that you
will use in your study. Please, refer to the variables by their name in
the codebook of the dataset that you are requesting data from. For
example, if you want to request data on participants' ages from the
SIGMA study, you need to list the variable as ``agl\_ado\_age'', not
``participant age''. The codebooks are available at . If you are
requesting data from multiple measurement points, please specify which
variables you need for every measurement point. Similarly, clearly
mention if you only need a subset of the data, e.g.~for power
calculations.

You will also need to specify the format in which you want to receive
the data. R, Stata, SAS and SPSS users can request a .csv file with code
to label the data. Users of other statistical software can request a
.csv file with only the raw data, a .csv file with labels or a CDISC
ODM-XML file. You can only choose one format.

After completing the request form, you can download a date and
time-stamped overview of your variable access request. We recommend
downloading this overview, so that you can compare your requested
variables with the variables that you will receive from the data
manager.

\textbf{Important to note:} after submitting the variable access
request, it is no longer possible to make any changes to the list of
requested variables. This includes adding and removing variables from
the list. So make sure in advance that you have listed all variables
that you will need for your study. Do not request variables that are not
mentioned in the post-registration, you will not get access to them.

\hypertarget{receiving-data}{%
\subsection{Receiving data}\label{receiving-data}}

After your variable access request is approved, you will receive an
automatic email notifying you of the approval. Shortly after, the data
manager will send you the dataset in the requested format. Along with
the dataset, you will receive an email with the following message:

``The data requested by {[}your name{]} for the study titled {[}your
study title{]} have been sent on {[}date{]}''

Please check the dataset immediately after receiving it. Make sure that
all variables you requested are included, that you received data from
the correct measurement point(s) and that you received data from the
correct subset of participants when applicable. You can use the download
of the variable request form to check if everything is correct. In case
there is a mistake in the dataset, contact the data manager as soon as
possible. We will make sure that you receive the correct dataset.

\hypertarget{section-4}{%
\subsection{}\label{section-4}}

\hypertarget{special-cases}{%
\subsection{Special cases}\label{special-cases}}

\hypertarget{section-5}{%
\subsubsection{}\label{section-5}}

\hypertarget{registered-reports-using-drops}{%
\subsubsection{Registered Reports using
DROPS}\label{registered-reports-using-drops}}

\hypertarget{about-registered-reports}{%
\paragraph{About Registered Reports}\label{about-registered-reports}}

Registered Reports are a novel publication format with a two-stage
research process. In the first stage, researchers are asked to specify
their introduction, hypotheses, methods and analyses before collecting
any data. These are then reviewed based on their theoretical and
methodological soundness. When the paper is approved in the first stage,
it is awarded an \emph{in-principle acceptance}, which guarantees that
the paper will be published after the data collection and analysis are
completed, regardless of what the results are. In stage 2 of the review
process, the reviewers make sure that the research that was conducted
was in line with the protocol registered in Stage 1. In this way, the
Registered Reports format prevents publication bias (i.e., ensures that
a paper will be judged by its methods, not by its results) and gives
researchers the opportunity to fine-tune their protocol before they
collect data. We highly recommend considering the Registered Reports
format for your secondary data analysis. For more information about the
format itself and the review process, we recommend
\href{https://www.nature.com/articles/s41562-021-01193-7}{this paper}.

\includegraphics{RackMultipart20221220-1-9i7noz_html_26ada00dbe266d99.png}

\emph{Figure 2: The Registered Reports workflow (reproduced from}
\href{https://www.cos.io/initiatives/registered-reports}{\emph{https://www.cos.io/initiatives/registered-reports}}\emph{)}

\hypertarget{registered-reports-and-drops}{%
\paragraph{Registered Reports and
DROPS}\label{registered-reports-and-drops}}

If you decide to write an article using the
\href{https://www.cos.io/initiatives/registered-reports}{Registered}\href{https://www.cos.io/initiatives/registered-reports}{Reports
format}\href{https://www.zotero.org/google-docs/?UK8wKf}{(Chambers,
20}\href{https://www.zotero.org/google-docs/?UK8wKf}{13}\href{https://www.zotero.org/google-docs/?UK8wKf}{;
Nosek \& Lakens, 2014)}, the process of requesting data via DROPS is
similar to the standard procedure. As the first step, you submit an
abstract in DROPS. After the abstract is approved, you will receive an
automated email from DROPS asking you to submit the variable request
form. In the variable request form, please indicate that the
registration of your study is not yet available, as you are waiting for
the Stage 1 manuscript acceptance.

Likely, you will need a subsample of the data for power calculations
before writing your Stage 1 manuscript and study registration. If that
is the case, it is possible to submit a first abstract and variable
request form in DROPS before submitting your introduction and methods
section as a Stage 1 Registered Report. If you plan to do so, please
state clearly that you are only requesting a subsample of the data. In
the acceptance email for your abstract submission, reference number will
be included that you will use to request the remaining data after your
Stage 1 manuscript is accepted.

After your Stage-1 manuscript is accepted by the journal, please submit
a new abstract to request the remaining data for your main analyses.
Then, you will be asked to submit your study registration and (second)
variable request form to DROPS. After the variable request form is
approved by DROPS, you will be able to download a time-stamped overview
that confirms you have only accessed the dataset after Stage-1
manuscript acceptance.

\hypertarget{exploratory-studies}{%
\subsection{Exploratory studies}\label{exploratory-studies}}

It is possible to request DROPS datasets to conduct exploratory data
analysis. Of course, given that some of the rules and suggestions
mentioned above are aimed mostly at confirmatory approaches, they do not
apply to exploratory analyses. Still, researchers should aim at maximum
transparency when planning and reporting their exploratory analyses.

While you do not have to register explicit hypotheses or plan for every
possible contingency, we recommend that you clearly state the general
intent of the study (what are you starting points, what do you want to
learn), clarify your choice of variables and pre-processing/analytical
plans. During the analysis itself, we recommend you to log the different
analyses and decisions you conducted in a transparent way. For example,
you can add updates to your original registration, explaining what
variables and analytical procedures you added and why.

For more information about pre-registering exploratory studies (and
pre-registering secondary data analyses in general), we recommend
\href{https://open.lnu.se/index.php/metapsychology/article/view/2625}{this
article}\href{https://www.zotero.org/google-docs/?osv6BM}{(Van den Akker
et al., 2021)}. Additionally, the
\href{https://www.elsevier.com/__data/promis_misc/Exploratory_Reports_Guidelines.pdf}{guidelines
for the Exploratory Reports publication format} are a good starting
point for transparent exploratory secondary data analysis. Specifically,
we recommend reporting effect sizes, confidence/credible intervals, or
just directionalities of effects, instead of \emph{p}-values in
exploratory papers
\href{https://www.zotero.org/google-docs/?NsZXQE}{(Weston et al.,
2019)}.

\hypertarget{section-6}{%
\section{}\label{section-6}}

\hypertarget{frequently-asked-questions}{%
\section{Frequently Asked Questions}\label{frequently-asked-questions}}

\begin{itemize}
\tightlist
\item
  Q: \emph{How long does it approximately take to receive data through
  DROPS?}

  \begin{itemize}
  \tightlist
  \item
    If you do not receive a response within 2 weeks of submitting your
    abstract or you do not receive the data within 2 weeks from
    submitting the data request form, please reach out to the
    \protect\hyperlink{_6eanlztnrvpa}{contact person} of the respective
    study.
  \end{itemize}
\item
  Q: \emph{What if my pre-registered models do not converge / there are
  errors in the pre-registered code / I decided to use different
  analyses than the ones I registered / assumptions of the statistical
  models were violated?}

  \begin{itemize}
  \tightlist
  \item
    A: Some of the decisions you will make during the registration
    process will be dependent on the data. For example, upon finding out
    that your data violates assumptions you might want to use a
    statistical model different from the one you were initially planning
    to use. In the case of multilevel models widely used to analyse ESM
    data, non-convergence is a widespread issue. Although it is usually
    not possible to predict all possible issues that can arise during
    data analysis, we encourage you to include detailed contingency
    plans in your study registration. For example, you can pre-specify
    how you are going to deal with convergence issues or violated
    assumptions (and include the code for these contingencies in the
    registration).
  \end{itemize}
\end{itemize}

Still, unexpected problems can arise even if you rigorously plan for
possible contingencies. In that case, we recommend you to add updates
reflecting the necessary changes to your registration and to clearly
communicate the deviations from the registered analysis in the paper
itself. The goal of study registration is maximum transparency, not
perfection. For more information about contingency plans and reporting
deviations from registration, we recommend the ESM pre-registration
tutorial \href{https://www.zotero.org/google-docs/?A6JUag}{(Kirtley,
Lafit, et al., 2021)}.

Please note that unless the deviations from registration requires
accessing other data, it is not necessary to submit a new DROPS abstract
with details of the new registration/deviation from the registration.

\begin{itemize}
\tightlist
\item
  \emph{Q: I know broadly what variables I want to access but I have
  some questions. Can I use DROPS to ask questions?}

  \begin{itemize}
  \tightlist
  \item
    A: Instead of using the DROPS forms to ask questions, please send
    your questions via email to the DROPS data manager
    \href{https://www.kuleuven.be/wieiswie/nl/person/00007576}{Dr
    Martien Wampers}.
  \end{itemize}
\item
  Q: \emph{I am a reviewer / I am planning to do a verification report
  of a study that used one of the datasets included in DROPS. To obtain
  the data for reanalysis, do I need to submit my own DROPS request
  (complete with abstract etc.), or is there a faster way to obtain the
  exact dataset that the original authors used?}

  \begin{itemize}
  \tightlist
  \item
    As a reviewer, we recommend you to reach out directly to the contact
    person of the given dataset (see the
    \protect\hyperlink{_6eanlztnrvpa}{\emph{Which data can be request
    through DROPS?}}subsection) and see if you can obtain the dataset
    based on the paper title.
  \end{itemize}
\item
  Q: \emph{I want to first request 70\% of the sample to create a model,
  then, after I have the final model, I will request the remaining 30\%
  to cross-validate it. Do I need to submit two separate requests?}

  \begin{itemize}
  \tightlist
  \item
    A: Yes, DROPS allows for requesting subsamples of the datasets for
    the purposes of cross-validation. Please refer to the section about
    \protect\hyperlink{_qlx2etgrq9hr}{requesting a subsample for power
    analysis}.
  \end{itemize}
\item
  Q: \emph{The journal I am submitting to asks that my data are made
  publicly available, what do I do?}

  \begin{itemize}
  \tightlist
  \item
    You can state that the you used are stored in a restricted-access
    data repository maintained by the Center for Contextual Psychiatry
    and mention that there is a data-management system in place.
  \end{itemize}
\item
  Q: \emph{What do I write in the `data availability' statement when I
  submit my manuscript to a journal for publication?}

  \begin{itemize}
  \tightlist
  \item
    See response above
  \end{itemize}
\item
  Q: \emph{I am a supervisor and one of my students is going to use data
  from DROPS for their thesis, but I also want to write a paper on these
  data later on. Do we need two post-registrations?}

  \begin{itemize}
  \tightlist
  \item
    If you are planning to run two separate studies, two separate
    post-registrations will be required.
  \end{itemize}
\item
  Q: \emph{One of my colleagues/students is going to run analyses on the
  same variables I used in a previous paper. Can I just send them the
  data to save time?}

  \begin{itemize}
  \tightlist
  \item
    A: Although we understand that just sharing the data with your
    colleague might be the more practical way of doing things, one of
    the main functions of DROPS is to keep track of who accessed the
    datasets. This makes the lives of researchers easier as well, as
    DROPS provides them with date- and time-stamped confirmation of data
    access, that can be used to prove the data were only accessed after
    the study registration was submitted. As such, please start a new
    DROPS request for every study you are planning to conduct, even if
    it uses identical variables to a study you've already conducted.
  \end{itemize}
\item
  Q: \emph{How do I know if there is missing data in some of the
  variables I would like to use?}

  \begin{itemize}
  \tightlist
  \item
    The general information about missing data and compliance in ESM
    variables are usually included in the study protocols.
  \end{itemize}
\item
  Q: \emph{I don't know anything about pre-registration or open science.
  Help! Where do I start?}

  \begin{itemize}
  \tightlist
  \item
    There are many good-quality introductory resources for getting into
    open science. For starters, we would recommend the following two
    articles:

    \begin{itemize}
    \tightlist
    \item
      \href{https://econtent.hogrefe.com/doi/full/10.1027/2151-2604/a000387}{Seven
      Easy Steps to Open
      Science}\href{https://www.zotero.org/google-docs/?t8Vk4A}{(Crüwell
      et al., 2019)}
    \item
      Easing Into Open Science: A Guide for Graduate Students and their
      Advisors
      \href{https://www.zotero.org/google-docs/?ho5j2d}{(Kathawalla et
      al., 2021)}
    \end{itemize}
  \end{itemize}
\item
  Q: \emph{What formats can I request DROPS datasets in?}

  \begin{itemize}
  \tightlist
  \item
    A: The two main formats you can request DROPS data in are csv and
    CDISC ODM-XML. For more information, please refer to
    \protect\hyperlink{_2dzt9xeysbee}{this section of the manual}.
  \end{itemize}
\item
  Q: \emph{I provided a view-only link to the entire OSF project with
  the registration, can you see it?}

  \begin{itemize}
  \tightlist
  \item
    A: Unfortunately, the only way we can access your registration is
    via a view only link to \textbf{the registration itself}.
  \end{itemize}
\item
  Q: \emph{Can I get access to an entire dataset?}

  \begin{itemize}
  \tightlist
  \item
    A: To be consistent with the main goals of the DROPS data curation
    system, we require researchers to only request variables that are
    directly relevant to your planned study.
  \end{itemize}
\end{itemize}

\textbf{Bibliography}

\href{https://www.zotero.org/google-docs/?EJLlNC}{Benning, S. D.,
Bachrach, R. L., Smith, E. A., Freeman, A. J., \& Wright, A. G. (2019).
The registration continuum in clinical science: A guide toward
transparent
practices.}\href{https://www.zotero.org/google-docs/?EJLlNC}{\emph{Journal
of Abnormal
Psychology}}\href{https://www.zotero.org/google-docs/?EJLlNC}{,}\href{https://www.zotero.org/google-docs/?EJLlNC}{\emph{128}}\href{https://www.zotero.org/google-docs/?EJLlNC}{(6),
528.}

\href{https://www.zotero.org/google-docs/?EJLlNC}{Chambers, C. D.
(2020). Verification Reports: A new article type at
Cortex.}\href{https://www.zotero.org/google-docs/?EJLlNC}{\emph{Cortex}}\href{https://www.zotero.org/google-docs/?EJLlNC}{,}\href{https://www.zotero.org/google-docs/?EJLlNC}{\emph{129}}\href{https://www.zotero.org/google-docs/?EJLlNC}{,
A1--A3. https://doi.org/10.1016/j.cortex.2020.04.020}

\href{https://www.zotero.org/google-docs/?EJLlNC}{Champely, S., Ekstrom,
C., Dalgaard, P., Gill, J., Weibelzahl, S., Anandkumar, A., Ford, C.,
Volcic, R., De Rosario, H., \& De Rosario, M. H. (2018). Package
`pwr.'}\href{https://www.zotero.org/google-docs/?EJLlNC}{\emph{R Package
Version}}\href{https://www.zotero.org/google-docs/?EJLlNC}{,}\href{https://www.zotero.org/google-docs/?EJLlNC}{\emph{1}}\href{https://www.zotero.org/google-docs/?EJLlNC}{(2).}

\href{https://www.zotero.org/google-docs/?EJLlNC}{Cohen, J. (1992).
Statistical power
analysis.}\href{https://www.zotero.org/google-docs/?EJLlNC}{\emph{Current
Directions in Psychological
Science}}\href{https://www.zotero.org/google-docs/?EJLlNC}{,}\href{https://www.zotero.org/google-docs/?EJLlNC}{\emph{1}}\href{https://www.zotero.org/google-docs/?EJLlNC}{(3),
98--101.}

\href{https://www.zotero.org/google-docs/?EJLlNC}{Crüwell, S., van
Doorn, J., Etz, A., Makel, M. C., Moshontz, H., Niebaum, J. C., Orben,
A., Parsons, S., \& Schulte-Mecklenbeck, M. (2019). Seven Easy Steps to
Open Science: An Annotated Reading
List.}\href{https://www.zotero.org/google-docs/?EJLlNC}{\emph{Zeitschrift
Für
Psychologie}}\href{https://www.zotero.org/google-docs/?EJLlNC}{,}\href{https://www.zotero.org/google-docs/?EJLlNC}{\emph{227}}\href{https://www.zotero.org/google-docs/?EJLlNC}{(4),
237--248. https://doi.org/10.1027/2151-2604/a000387}

\href{https://www.zotero.org/google-docs/?EJLlNC}{Eisele, G., Lafit, G.,
Vachon, H., Kuppens, P., Houben, M., Myin-Germeys, I., \& Viechtbauer,
W. (2021). Affective structure, measurement invariance, and reliability
across different experience sampling
protocols.}\href{https://www.zotero.org/google-docs/?EJLlNC}{\emph{Journal
of Research in
Personality}}\href{https://www.zotero.org/google-docs/?EJLlNC}{,}\href{https://www.zotero.org/google-docs/?EJLlNC}{\emph{92}}\href{https://www.zotero.org/google-docs/?EJLlNC}{,
104094. https://doi.org/10.1016/j.jrp.2021.104094}

\href{https://www.zotero.org/google-docs/?EJLlNC}{Eisele, G., Vachon,
H., Lafit, G., Kuppens, P., Houben, M., Myin-Germeys, I., \&
Viechtbauer, W. (2020). The Effects of Sampling Frequency and
Questionnaire Length on Perceived Burden, Compliance, and Careless
Responding in Experience Sampling Data in a Student
Population.}\href{https://www.zotero.org/google-docs/?EJLlNC}{\emph{Assessment}}\href{https://www.zotero.org/google-docs/?EJLlNC}{,}\href{https://www.zotero.org/google-docs/?EJLlNC}{\emph{29(2)}}\href{https://www.zotero.org/google-docs/?EJLlNC}{,
16.}

\href{https://www.zotero.org/google-docs/?EJLlNC}{Faul, F., Erdfelder,
E., Buchner, A., \& Lang, A.-G. (2009). Statistical power analyses using
G* Power 3.1: Tests for correlation and regression
analyses.}\href{https://www.zotero.org/google-docs/?EJLlNC}{\emph{Behavior
Research
Methods}}\href{https://www.zotero.org/google-docs/?EJLlNC}{,}\href{https://www.zotero.org/google-docs/?EJLlNC}{\emph{41}}\href{https://www.zotero.org/google-docs/?EJLlNC}{(4),
1149--1160.}

\href{https://www.zotero.org/google-docs/?EJLlNC}{Kathawalla, U.-K.,
Silverstein, P., \& Syed, M. (2021). Easing Into Open Science: A Guide
for Graduate Students and Their
Advisors.}\href{https://www.zotero.org/google-docs/?EJLlNC}{\emph{Collabra:
Psychology}}\href{https://www.zotero.org/google-docs/?EJLlNC}{,}\href{https://www.zotero.org/google-docs/?EJLlNC}{\emph{7}}\href{https://www.zotero.org/google-docs/?EJLlNC}{(1),
18684. https://doi.org/10.1525/collabra.18684}

\href{https://www.zotero.org/google-docs/?EJLlNC}{Kirtley, O. J.,
Achterhof, R., Hagemann, N., Hermans, K. S. F. M., Hiekkaranta, A. P.,
Lecei, A., Boets, B., Henquet, C., Kasanova, Z., Schneider, M., van
Winkel, R., Reininghaus, U., Viechtbauer, W., \& Myin-Germeys, I.
(2021).}\href{https://www.zotero.org/google-docs/?EJLlNC}{\emph{Initial
cohort characteristics and protocol for SIGMA: An accelerated
longitudinal study of environmental factors, inter- and intrapersonal
processes, and mental health in
adolescence}}\href{https://www.zotero.org/google-docs/?EJLlNC}{{[}Preprint{]}.
PsyArXiv. https://doi.org/10.31234/osf.io/jp2fk}

\href{https://www.zotero.org/google-docs/?EJLlNC}{Kirtley, O. J., Lafit,
G., Achterhof, R., Hiekkaranta, A. P., \& Myin-Germeys, I. (2021).
Making the Black Box Transparent: A Template and Tutorial for
Registration of Studies Using Experience-Sampling
Methods.}\href{https://www.zotero.org/google-docs/?EJLlNC}{\emph{Advances
in Methods and Practices in Psychological
Science}}\href{https://www.zotero.org/google-docs/?EJLlNC}{,}\href{https://www.zotero.org/google-docs/?EJLlNC}{\emph{4}}\href{https://www.zotero.org/google-docs/?EJLlNC}{(1),
251524592092468. https://doi.org/10.1177/2515245920924686}

\href{https://www.zotero.org/google-docs/?EJLlNC}{Lafit, G., Adolf, J.,
Dejonckheere, E., Myin-Germeys, I., Viechtbauer, W., \& Ceulemans, E.
(2020).}\href{https://www.zotero.org/google-docs/?EJLlNC}{\emph{Selection
of the Number of Participants in Intensive Longitudinal Studies: A
User-friendly Shiny App and Tutorial to Perform Power Analysis in
Multilevel Regression Models that Account for Temporal
Dependencies}}\href{https://www.zotero.org/google-docs/?EJLlNC}{{[}Preprint{]}.
PsyArXiv. https://doi.org/10.31234/osf.io/dq6ky}

\href{https://www.zotero.org/google-docs/?EJLlNC}{Nosek, B. A.,
Ebersole, C. R., DeHaven, A. C., \& Mellor, D. T. (2018). The
preregistration
revolution.}\href{https://www.zotero.org/google-docs/?EJLlNC}{\emph{Proceedings
of the National Academy of
Sciences}}\href{https://www.zotero.org/google-docs/?EJLlNC}{,}\href{https://www.zotero.org/google-docs/?EJLlNC}{\emph{115}}\href{https://www.zotero.org/google-docs/?EJLlNC}{(11),
2600--2606. https://doi.org/10.1073/pnas.1708274114}

\href{https://www.zotero.org/google-docs/?EJLlNC}{Nosek, B. A., \&
Lakens, D.
(2014).}\href{https://www.zotero.org/google-docs/?EJLlNC}{\emph{Registered
reports: A method to increase the credibility of published results.}}

\href{https://www.zotero.org/google-docs/?EJLlNC}{Perugini, M.,
Gallucci, M., \& Costantini, G. (2018). A practical primer to power
analysis for simple experimental
designs.}\href{https://www.zotero.org/google-docs/?EJLlNC}{\emph{International
Review of Social
Psychology}}\href{https://www.zotero.org/google-docs/?EJLlNC}{,}\href{https://www.zotero.org/google-docs/?EJLlNC}{\emph{31}}\href{https://www.zotero.org/google-docs/?EJLlNC}{(1).}

\href{https://www.zotero.org/google-docs/?EJLlNC}{Qiu, W.
(2021).}\href{https://www.zotero.org/google-docs/?EJLlNC}{\emph{powerMediation:
Power/Sample Size Calculation for Mediation
Analysis}}\href{https://www.zotero.org/google-docs/?EJLlNC}{(0.3.4).}

\href{https://www.zotero.org/google-docs/?EJLlNC}{Trull, T. J., \&
Ebner-Priemer, U. W.
(2020).}\href{https://www.zotero.org/google-docs/?EJLlNC}{\emph{Ambulatory
Assessment in Psychopathology Research: A Review of Recommended
Reporting Guidelines and Current
Practices}}\href{https://www.zotero.org/google-docs/?EJLlNC}{. 8.}

\href{https://www.zotero.org/google-docs/?EJLlNC}{Van den Akker, O. R.,
Weston, S., Campbell, L., Chopik, B., Damian, R., Davis-Kean, P., Hall,
A., Kosie, J., Kruse, E., Olsen, J., Ritchie, S., Valentine, K., Van 't
Veer, A., \& Bakker, M. (2021). Preregistration of secondary data
analysis: A template and
tutorial.}\href{https://www.zotero.org/google-docs/?EJLlNC}{\emph{Meta-Psychology}}\href{https://www.zotero.org/google-docs/?EJLlNC}{,}\href{https://www.zotero.org/google-docs/?EJLlNC}{\emph{5}}\href{https://www.zotero.org/google-docs/?EJLlNC}{.
https://doi.org/10.15626/MP.2020.2625}

\href{https://www.zotero.org/google-docs/?EJLlNC}{Weston, S. J.,
Ritchie, S. J., Rohrer, J. M., \& Przybylski, A. K.
(2019).}\href{https://www.zotero.org/google-docs/?EJLlNC}{\emph{Recommendations
for Increasing the Transparency of Analysis of Preexisting Data
Sets}}\href{https://www.zotero.org/google-docs/?EJLlNC}{. 14.}

\end{document}
